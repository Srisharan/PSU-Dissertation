\chapter{Look File Format Description and Biaxtools}

\section{Purpose}
This script was designed to read data into an easy to use format from xlook output in ASCII or binary form.  Data is read, the header parsed for column names, lengths, etc., and a rec array or Pandas dataframe object returned that is easy to access and call from within a script.  Column names are used to call the data columns, so as long as consistent naming is used the column order in the file is irrelevant.

\section{Description}
\subsection{ASCII Files}
The function ReadAscii first opens the text file with a standard open command.  We know that each column in the data is written as 12 characters wide.  The first line of the header is the number of records, the second is the column number, the third the column headings, the fourth the column units, and the fifth the number of records in each column.  This information if parsed and stored.  Numerical data is read into an array.  A rec array is created with the numpy package and the labels of the column names that were parsed or the data is converted into a dataframe object. 

\subsection{Binary Files}
The function ReadBin opens the binary file in 'read binary' mode.  The initial file information is stored as follows (in big-endian format):


\begin{table}[h]
	\begin{center}
	\begin{tabular}{| l | l | l |}
		\hline
		Information & Format & Bytes\\
		\hline
		Name & 20 characters &  20\\
		\hline
		Number of Columns & int & 4 \\
		\hline
		Sweep & int & 4\\
		\hline
		Date/Time & int & 4\\
		\hline
	\end{tabular}
	\end{center}
	\label{BinaryFileHeadFormat}
\end{table}

For each possible column there are 84 bytes of information.  There are 32 possible columns.  We read the information for each column and store information for columns that have data.  Blank columns are identified by the first 6 characters of the name `no\_val'.  

\begin{table}[h]
	\begin{center}
	\begin{tabular}{| l | l | l |}
		\hline
		Information & Format & Bytes\\
		\hline
		Name & 13 characters &  13\\
		\hline
		Units & 13 characters &  13 \\
		\hline
		Gain & int & 4\\
		\hline
		Comment & 50 characters & 50\\
		\hline
		Number of Elements & int & 4\\
		\hline
	\end{tabular}
	\end{center}
	\label{BinaryColHeadFormat}
\end{table}

Data is stored as the default machine format, which for modern Intel based computers in little-endian.  This is the default for the module, but big-endian can be specified.  Each column is written out as a sequence of doubles that are $n_elem$ long.  Doubles are read and stored in a numpy array.  The array is then combined with datatype information collected from the headers and stored as a rec array.

To find the default machine binay format run: \emph{python -c "import struct; print 'little' if ord(struct.pack('L', 1)[0]) else 'big'"}

\section{Cautions}
A few cautions should be observed when using the biaxread script:

\begin{enumerate}
\item The empty array is shaped by row number information from the header.  If a datafile is cut, but the header is left unmodified there will be many extra zero data pairs at the end of the array.
\end{enumerate}

\section{Usage}

\subsection{Return Formats}
By default data will be returned as a Numpy record array (recarray).  Data can also be returned as a Pandas dataframe object that is indexed on the row number of the data.  To do this, set \emph{pandas=True} in the function call to either ReadAscii or ReadBin.

\subsection{ASCII Files}
First process the experimental data in xlook and output a text file with headers.  To do this input \emph{type 0 -1 1 12 pxxxx\_data.txt} at the xlook command line or add it to the data reduction file.  Be sure to correct any column or row number specifications here.  Start IPython or open your own Python script.  Import biaxread and pass the function ReadAscii the name of the text output file from xlook.  The array is now returned and ready to use.  


\subsection{Binary Files}
Process the experimental data in xlook and ouput a binary file.  This is done with the `write' command with the syntax \emph{write filename}.  In IPython or your Python script import biaxread and pass the function ReadBin the name of the binary file.  The array is returned and ready to use.

Files that were written in a big-endian format by setting dataendianness to 'big'.  The function call would look like \emph{ReadBin(filename,dataendianness='big')}.

\section{Acknowledgements}
Thank you to Marco Scuderi for using the code and being patient as little issues were worked out.  Please send any bug reports to kd5wxb@gmail.com or as an issue in the github repository.


\newpage

\section{Code}
\begin{lstlisting}

import numpy as np
import struct
import pandas as pd

def ReadAscii(filename,pandas=False):
    """
    Takes a filename containing the text output (with headers) from xlook and
    reads the columns into a rec array or dataframe object for easy data 
    processing and access.
    """

    try:
        f = open(filename,'r')
    except:
        print "Error Opening %s" %filename
        return 0
    
    col_width = 12 # Columns are 12 char wide in header
    
    # First line of the file is the number of records
    num_recs = f.readline()
    num_recs = int(num_recs.strip('number of records = '))
    print "\nNumber of records: %d" %num_recs
    
    # Second line is column numbers, we don't care so just count them
    num_cols = f.readline()
    num_cols = num_cols.split('col')
    num_cols = len(num_cols)
    print "Number of columns: %d" %num_cols
    
    # Third line is the column headings
    col_headings_str = f.readline()
    col_headings_str = col_headings_str[5:-1]
    col_headings = ['row_num'] # Row number the the first (unlabeled) column
    for i in xrange(len(col_headings_str)/12):
        heading = col_headings_str[12*i:12*i+12].strip()
        col_headings.append(heading)

    # Fourth line is column units
    col_units_str = f.readline()
    col_units_str = col_units_str[5:-1]
    col_units=[]
    for i in xrange(len(col_units_str)/12):
        heading = col_units_str[12*i:12*i+12].strip()
        col_units.append(heading)
    col_units = [x for x in col_units if x != '\n']  #Remove newlines
    
    # Fifth line is number of records per column
    col_recs = f.readline()
    col_recs = col_recs.split('recs')
    col_recs = [int(x) for x in col_recs if x != '\n']
    
    # Show column units and headings
    print "\n\n-------------------------------------------------"
    print "|%15s|%15s|%15s|" %('Name','Unit','Records')
    print "-------------------------------------------------"
    for column in zip(col_headings,col_units,col_recs):
        print "|%15s|%15s|%15s|" %(column[0],column[1],column[2])
    print "-------------------------------------------------"
    
    # Read the data into a numpy recarray
    dtype=[]
    #dtype.append(('row_num','float'))
    for name in col_headings:
        dtype.append((name,'float'))
    dtype = np.dtype(dtype)
    
    data = np.zeros([num_recs,num_cols])
    
    i=0
    for row in f:
        row_data = row.split()
        for j in xrange(num_cols):
            data[i,j] = row_data[j]
        i+=1

    f.close()
    
    if pandas==True: 
        # If a pandas object is requested, make a data frame 
        # indexed on row number and return it
        dfo  = pd.DataFrame(data,columns=col_headings)
        dfo = dfo.set_index('row_num') 
        return dfo
        
    else:  
        # Otherwise return the default (Numpy Recarray)  
        data_rec = np.rec.array(data,dtype=dtype)
        return data_rec

def ReadBin(filename,dataendianness='little',pandas=False):
    """
    Takes a filename containing the binary output from xlook and
    reads the columns into a rec array or dataframe object for easy 
    data processing and access.
    
    The data section of the file is written in the native format of the machine
    used to produce the file.  Endianness of data is little by default, but may
    be changed to 'big' to accomodate older files or files written on power pc 
    chips.
    """

    try:
        f = open(filename,'rb')
    except:
        print "Error Opening %s" %filename
        return 0
    
    col_headings = []
    col_recs     = []
    col_units    = []
    
    # Unpack information at the top of the file about the experiment
    name = struct.unpack('20c',f.read(20))
    name = ''.join(str(i) for i in name)
    name = name.split("\0")[0]
    print "\nName: ",name
    
    # The rest of the header information is written in big endian format
    
    # Number of records (int)
    num_recs = struct.unpack('>i',f.read(4))
    num_recs = int(num_recs[0])
    print "Number of records: %d" %num_recs
    
    # Number of columns (int)
    num_cols =  struct.unpack('>i',f.read(4))
    num_cols = int(num_cols[0])
    print "Number of columns: %d" %num_cols
    
    # Sweep (int) - No longer used
    swp =  struct.unpack('>i',f.read(4))[0]
    print "Swp: ",swp
    
    # Date/time(int) - No longer used
    dtime =  struct.unpack('>i',f.read(4))[0]
    print "dtime: ",dtime
    
    # For each possible column (32 maximum columns) unpack its header
    # information and store it.  Only store column headers of columns
    # that contain data.  Use termination at first NUL.
    for i in range(32):
    
        # Channel name (13 characters)
        chname = struct.unpack('13c',f.read(13))
        chname = ''.join(str(i) for i in chname)
        chname = chname.split("\0")[0] 
        
        # Channel units (13 characters)
        chunits = struct.unpack('13c',f.read(13))
        chunits = ''.join(str(i) for i in chunits)
        chunits = chunits.split("\0")[0]
    
        # This field is now unused, so we just read past it (int)
        gain = struct.unpack('>i',f.read(4))
        
        # This field is now unused, so we just read past it (50 characters)
        comment = struct.unpack('50c',f.read(50))

        # Number of elements (int)
        nelem = struct.unpack('>i',f.read(4))
        nelem = int(nelem[0])
        
        if chname[0:6] == 'no_val':
            continue # Skip Blank Channels
        else:
            col_headings.append(chname)
            col_recs.append(nelem)
            col_units.append(chunits)
    
    
    # Show column units and headings
    print "\n\n-------------------------------------------------"
    print "|%15s|%15s|%15s|" %('Name','Unit','Records')
    print "-------------------------------------------------"
    for column in zip(col_headings,col_units,col_recs):
        print "|%15s|%15s|%15s|" %(column[0],column[1],column[2])
    print "-------------------------------------------------"
    
    # Read the data into a numpy recarray
    dtype=[]
    for name in col_headings:
        dtype.append((name,'double'))
    dtype = np.dtype(dtype)
    
    data = np.zeros([num_recs,num_cols])
    
    for col in range(num_cols):
        for row in range(col_recs[col]):
            if dataendianness == 'little':
                data[row,col] = struct.unpack('<d',f.read(8))[0]
            elif dataendianness == 'big':
                data[row,col] = struct.unpack('>d',f.read(8))[0]
            else:
                print "Data endian setting invalid, please check and retry"
                return 0

    data_rec = np.rec.array(data,dtype=dtype)
    
    f.close()
    
    if pandas==True: 
        # If a pandas object is requested, make a data frame 
        # indexed on row number and return it
        dfo  = pd.DataFrame(data,columns=col_headings)
        # Binary didn't give us a row number, so we just let
        # pandas do that and name the index column
        dfo.index.name = 'row_num' 
        return dfo
        
    else:  
        # Otherwise return the default (Numpy Recarray)  
        data_rec = np.rec.array(data,dtype=dtype)
        return data_rec

\end{lstlisting}