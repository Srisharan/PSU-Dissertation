%% thesis.tex
%%
%% this file, mythesis.tex, is the main file of a fictitious
%% Penn State Ph D thesis 
%%
%% 
%% this material can be used as a template to prepare your own Ph D thesis
%%
%% this file was created Sept 1995 by Stephen G. Simpson,
%% simpson@math.psu.edu
%%
%% revised November 1996, S. Simpson
%% revised 2002, Sarah Gallager (to allow deluxetables)
%% modified a little more by Michele Stark (2004)
%% modified to use the new psuthesis.cls (Mar 2005) with a signature 
%%   page and a committee page
%%   
%% modified again by Sonny Harman to reflect both MS and PhD theses

\documentclass[11pt]{psuthesis}

%% optional packages, in case you want AMS math macros and AMS symbols
\usepackage{amsmath,amssymb}
%% allows bibtex, \citet{}, \citep{} referencing:
\usepackage[square]{natbib}
\usepackage{lipsum}
%%Comment out the above package when you add your own chapters. Unless you want Lorem Ipsum.
%\usepackage[table]{xcolor}
%% I truthfully don't know what the following is for, but I never used it:
%\citestyle{aa}
%% optional package, in case you want PostScript graphics:
%\usepackage{psfig,graphics} 
%% the following allows you to use the AAS deluxetable environment:
%\usepackage{deluxetable}
%% Use the following if you have tables that are longer than a single page::
%\usepackage{longtable}
%% Use the following if you have (non-deluxetables, i.e., longtables) that
%% you need to display landscape oriented:
%\usepackage{lscape}

%% for a less-than-final version of the thesis, this command
%% places "DRAFT: <date> AT <time>" at the top of each page...
%\thesisdraft    
%% (comment this out for the final version)

%% you can speed things up by compiling only one chapter at a time
%\includeonly{somechapter}
%\includeonly{someotherchapter}
%\includeonly{yetanotherchapter}
%\includeonly{Ithinkyougettheideachapter}
%\includeonly{conclusions}
%% (comment this out for the final version)

%% Fix the text citations so that there is no comma between the authors and 
%% year.  This will help contain the furious Brandt red pen.
%\bibpunct{(}{)}{;}{a}{}{,}
\bibpunct{[}{]}{;}{a}{,}{,}

\renewcommand{\thesubsubsection}{(\roman{subsubsection})}


% usage - \mark{myfootnote}{This is my footnote} and 
% \recall{myfootnote}

%this defines where your images for the build are stored.
%\graphicspath{{images/}}

%% Change the fonts back to something reasonable.
%% Note: scriptsize is typically smaller than footnotesize.
%\renewcommand{\scriptsize{\@setfontsize\scriptsize\@ixpt{9pt}}
%\renewcommand{\footnotesize{\@setfontsize\scriptsize\@xpt{10pt}}

%%\thesisdraft
%% Uncomment the above line to generate a copy of the thesis that has the phrase:
%%  "Draft: <current date> at <current time>" to be printed in the head of each page.

%%
%% These are all the definitions that I've used throughout my thesis.
%\input{/enter/path/to/your/definitions}

%% This is just to set your degree. Set the toggle within the document (about 20 lines below).
\newtoggle{masters}


\begin{document}

%%Comment this out for the final thesis:

%%

%% at the beginning of the thesis we have a title page, a signature
%% page, and an abstract

\author{Name here}


\title{\uppercase {Title of your \\
	Dissertation }}
\dept{College of Earth and Mineral Sciences}
\major{Geosciences}

%% Just comment out the statement that isn't true -- if you're getting your MS, then 
%\toggletrue{masters}
%otherwise, uncomment this to switch to PhD mode.
\togglefalse{masters}

\submitdate{Month 20XX}

\copyrightyear{20XX}

%
%\begin{singlespace}

\readerone{Advisor \\
         \prof{Geosciences} \\
         \adviser \\
         %\chair
}

\readertwo{Committee Member\\
           \prof{Geosciences}
}

\readerthree{Committee Member\\
             \prof{Geosciences}
}

\readerfour{Administrator or Dept. Head  \\
	  \prof{Geosciences}
} 
%             Associate \head{Graduate Programs}}
%
%\end{singlespace}
%\readerfive{Evan Pugh Prof.\ Member4 \\
%             Evan Pugh \prof{Astronomy and Astrophysics}}
%
%\readersix{Assistant Prof.\ Member5 \\
%             \assistprof{Astronomy and Astrophysics}}
%
%\readerseven{Lee R. Kump \\
%             \prof{Geosciences}\\
%             \head{Geosciences}}

%%   Key to titles:
%% Associate Professor: \asocprof{of what}
%% Assistant Professor: \assistprof{of what}
%% Full Professor: \prof{of what}
%% also Dept. Head: \head{of what}
%% You can also do things like: ``Associate \head{of what}''

\begin{frontmatter}

%this is the ``normal'' signature page from the original version of the class - the official copy of the thesis or dissertation does not contain signatures of committee members, so omit the signature page. The committee page is a necessary evil.
%\signaturepage

\begin{doublespace}
\titlepage
\end{doublespace}

%this is the new committee page
\committeepage

\abstract

Here is where the text of the Abstract goes.



%% this is the end of the abstract

%% after the abstract come the table of contents, the list of tables,
%% and the list of figures
%% Note about the figure list... so the figure list is a decent length,
%% use the following command for the figure captions:
%% \caption[Short figure title to appear in figure list]{Normal figure caption}
%% (this trick unfortunately does not work with the deluxetable 
%% ``\tablecaption'' command, so be careful what you put in the table captions)
%%
%% If you have really long tables that cover multiple pages, you
%% will want to use the ``longtable'' environment (it is very similer to
%% deluxetable) but allows you to specify headers and footers for the first,
%% last, and middle pages of the figure.

%% use this command to omit the list of tables,
%% if the thesis doesn't contain any tables
%\nolotables

%% use this command to omit the list of figures,
%% if the thesis doesn't contain any figures
%\nolofigures

\tables

%% next come the acknowlegements (optional) and the preface (optional)

\acknowledgments  % optional
%\begin{center}{
% To my family.} %, who have given much to see me this far, and to my wife, for putting up with me. 
%\end{center}

Here is where the acknowledgments go if you have any.


%\preface    % optional

\clearpage

\vspace*{2.0truein}

%\LARGE
%\parbox{4.0truein}{
%\par\noindent
%Rome did not create a great empire by having meetings, they did it by
%killing all who opposed them.\\
%\hspace*{\fill}--Unknown
%}
%\normalsize
%\vspace{30pt}
%
%\LARGE
\parbox{4.0truein}{
\par\noindent
To steal ideas from one person is plagiarism;\\ to steal from many is research.\\
\hspace*{\fill}--Unknown
}
%\normalsize
\end{frontmatter}

%% this is the end of the front matter






%% now we include the actual chapters of the thesis
%% there are individual chapter files ch-intr.tex, ch-over.tex, ...
%% (NOTE: you do not need the ``.tex'' extention on the file name 
%% in the include statement)
%% these chapters can be in a sub-directory, for example: 
%% \include{chapterdirectory/chaptername}
%\chapter*{Introduction}

The movement of rocks relative to each other caused by tectonic strain accumulation results in zones of localized deformation. Strain energy is effectively released as slip, which is most efficiently accommodated in a relatively thin zone of movement oriented favorably with the regional stress field. These zones of slip host a variety of complex processes which are still poorly understood. The material properties of the rocks, and therefore the mineralogy, controls the frictional properties of the interface. Geochemical processes can move minerals around through dissolution-precipitation processes and even accommodate slow creep movement of the rock. Hydrologic properties of the system can serve to increase or decrease the fluid pressure on the fault and even determine if mechanisms such as dilation hardening are allowed to operate. The existing geologic system can introduce complications through many processes, including the roughening of the fault surface through subduction of seamounts and other structural processes. These slowly evolving geologic systems are transformed during seismic slip of meters/second into systems with rapid change and extreme conditions which are poorly understood. The transformation between the two is quite possibly the most important part of the process and is not able to be explored in the real world due to the distance and measurement limitations imposed by the geology.

Laboratory rock mechanics studies have been recreating portions of fault zone conditions and behavior on a very small scale (millimeters to meters), but have not been widely utilized by the broader earthquake community. Seismologists often have data that is spatially limited by the location of measured events and cannot resolve the details that are observed in the laboratory. Experimentalists are often focused on the minutia of the frictional dynamics of the system and are not able to connect their observations to the large-scale global experiment of plate tectonics. This disconnect calls for the realization that rock mechanics could and should be treated as experimental seismology. In this dissertation, I attempt to study the general mechanisms of slip mode determination and connect them to a single natural system. While that is a limited scope, I believe that more of such work can begin to build the bridge between the experimental and observational fields that is desperately needed to understand this complex phenomena.

Friction, or the force resisting the movement of two surfaces relative to each other, is one of the key parameters that control the nucleation stage of earthquakes. In the case of a two-block slider model, friction is defined as the ratio of shear stress to normal stress. Early models of friction used a single value of friction for materials at all velocities and conditions. For most Earth materials, this number is in the range 0.6-0.8. Early explorations of friction began to discover that friction is not a static quantity, but that it was different when stationary versus when in motion. These so-called static and dynamic friction values are often as complicated as many analyses get, but even they do not explain all the observations of careful experimentation. Observations of the velocity dependence of friction, time dependence of friction, and memory effects from past states all indicated the need for a more complex formulation. The rate-and-state friction framework captures many of these observations, but is at its core an empirical relation with arbitrary frictional parameters.

The rate-and-state friction relation allows friction to change as a function of the sliding velocity of the system (rate) and the history of the system (state). The parameters in the basic relation are the direct effect, $a$, the evolution effect, $b$, and the critical slip distance $D_c$ which are fully described in chapter 1. The state of the system is determined by a second equation, the state relation. Common state relations have the state as a function of either time or slip on the fault. This simple model can capture behaviors such as velocity strengthening/weakening, time dependent healing, and even a transition in behavior from stable to unstable. In recent years, there have been works which extend the basic model, from the use of multiple state variables and designer friction relations in which the frictional parameters are themselves functions of velocity, to more complex state relations.

Analysis of the rate-and-state friction relation shows that a Hopf bifurcation occurs under certain conditions and completely changes the behavior of the system. On one side of the bifurcation, the system is stable and responds to perturbations in a finite time and the system reaches a new state of equilibrium in which friction is constant. The other side of the bifurcation represents unstable behavior. In the unstable regime, the system never reaches a new equilibrium state and is continually experiencing accelerations. The sliding velocity of unstable systems is nearly zero for a period of time. During this stuck period the far field movement of the system accumulates energy stored as strain in the elastic media. Once the friction on the interface reaches a critical value, the system catastrophically fails and rapidly accelerates. During the slip phase of motion, heat is generated, material is pulverized, and elastic radiation is produced. This stick-slip cycle has often been cited as an analog for naturally occurring earthquakes, as well as a variety of other phenomena including cutting tool chatter and animal sound production.

In terms of the rate-and-state friction framework, the bifurcation is observed to occur at a value of stiffness of the system termed the critical stiffness $(k_c)$. When the system is stiffer than the critical stiffness, energy can be released from the system faster than the system can weaken, so it remains in a stable sliding state. When the system is more compliant than the critical stiffness, the system weakens faster than energy can be released, resulting in a growing energy imbalance that produces large accelerations and stick-slip behavior.  

While the strict bifurcation of behavior seemed to be adequate for earthquake science - faults either hosted earthquakes or did not - the engineering sciences were not satisfied. Examination of other dynamical systems shows that there are often regions of transitory or intermediate behavior. The boundary between the regions may indeed be sharp, but not infinitely sharp as predicted. For example, the behavior of control systems shows that regions of critical stability exist in which the system is not completely unstable and uncontrolled, but in which the controller is struggling to control the process variable. Likewise, with more comprehensive physics or complex relations, the rate-and-state friction relation can begin to exhibit more complex behaviors. 

The discovery of slow earthquakes confirmed that the frictional behavior of faults is not a simple stable-unstable boundary, but in fact a continuum that hosts a rich variety of complex behaviors. In this dissertation, I use small-scale laboratory studies to explore this spectrum of behaviors and determine which frictional properties and complications are needed to explain the observations. I apply these ideas to a natural slow-slipping system beneath Whillans Ice Stream in western Antarctica. My main contributions are demonstrations of how to generate the range of slip modes in the laboratory, to show that the critical stiffness ratio indeed explains many of the first order behaviors, and to show that the behavior of a system is indeed velocity dependent.

%\include{somechaper}
%\include{someotherchapter}
%\include{yetanotherchapter}
%
%
\chapter{Lorem Ipsum}
\lipsum[1]
\section{Section 1}
\lipsum[2-3]
\subsection{Subsection 1}
\lipsum[4-5]
\include{Chap2}
%
%
%\include{Ithinkyougettheideachapter}
%\begin{singlespace}
%\include{conclusions}

%% now we include the appendices
\appendices
%% omit this if there are no appendices

%% if there is only one appendix, 
%% say \singleappendix instead of \appendices
% \singleappendix

%% these are the actual commands to include the apendix files:

%\include{appendix1}

%\include{appendix2}



%% finally comes the bibliography and vita
%% the bibliography is generated automatically using BibTeX
%% Note: in order to get the (author year) citation style, this example 
%% includes a different *.bst style file than psuthesis.bst.  Sarah found 
%% this style file on: http://www.ee.oulu.fi/~harza/latex/  
%% This style includes the paper titles in the bibliography, so use it 
%% if you want:
%\bibliographystyle{/enter/path/to/ayphdthesis_mod}
%% However, according to the grad school thesis guide (c.2004), there is no 
%% special formating required for the bibliography, and to just follow 
%% the citation styles of your field, in that case the ``apj'' style is 
%% prefectly OK, so that is the one that is going to be included in this
%% example file (use whichever style you prefer):
\begin{singlespace}
\bibliographystyle{apa}
\end{singlespace}
%% the following is the path to you *.bib file 
%% (you do not need to enter the ``.bib'' extention)
\bibliography{yourbib}
%% ADS and Google Scholar can generate the entries in the bib file for you, just call up
%% the abstract, then near the bottom of the abstract page there is a
%% link to ``Bibtex entry for this abstract'' - just copy that into 
%% your bib file.  Here is an example bibtex entry:
%% @ARTICLE{citecode,
%%    author = {{Smith}, J. and {Jones}, M.},
%%     title = "{This Paper has some Really Cool Results}",
%%   journal = {\aj},
%%      year = 2002,
%%     month = sep,
%%    volume = 123,
%%     pages = {1-20},
%%    adsurl = {http://adsabs.harvard.edu/cgi-bin/nph-bib_query?bibcode=2002AJ....123.1S&amp;db_key=AST},
%%   adsnote = {Provided by the NASA Astrophysics Data System}
%% }
%% NOTE: ADS returns the AASTeX code for the journal name, you'll either 
%% have to change that by hand in each bib entry, or include a 
%% definition of the codes in your ``definitions'' file so LaTeX doesn't 
%% freak out.  Also, what I entered as ``citecode'' can be changed to 
%% whatever you want to use in the citations in the document, i.e., for
%% ``\citet{citecode}'' or ``\citep{citecode}''.


%% the thesis must end with a Curriculum Vitae (**one page or less**)
%% (this is Sarah's formatting, not sure how it compares to the other examples)
\iftoggle{masters}{\clearpage}{
\clearpage
\begin{singlespace}
\vita
\Large
\vspace*{-0.4truein}
\centerline{{\bf Name}}

\medskip

\large
\centerline{{\bf Education}}
\normalsize

\smallskip

\par\noindent
\textbf{\textit{The Pennsylvania State University}}\, State College, Pennsylvania\hfill 2011-Present

\smallskip

\par\noindent
\hspace{0.10truein}  
\parbox{6.15truein}{
\par\noindent
M.S. in Geosciences, expected in May 2014 }%\\ Area of Specialization: Planetary Atmospheres }

\medskip

\par\noindent
\textbf{\textit{Undergrad University}}\, Somewhere, Pennsylvania\hfill 2007-2010

\smallskip

\par\noindent
\hspace{0.10truein}  
\parbox{6.15truein}{
\par\noindent
B.S. in Science, minor in Mathematics, \textit{magna cum laude} with distinction in Science
}

\medskip

\large
\centerline{{\bf Awards and Honors}}
\normalsize

\smallskip

\par\noindent
An Award \hfill 20XX\\
Another Award \hfill 20XX\\
Something Cool \hfill 20XX\\

\medskip

\large
\centerline{{\bf Research Experience}}
\normalsize

%\smallskip

%\par\noindent
%\textbf{\textit{Doctoral Research}}\, The Pennsylvania State University\hfill199?--Present
%\par\noindent
%Thesis Advisor: Prof. Someone R. Other

%\smallskip
%
%\par\noindent
%\hspace{0.10truein}  
%\parbox{5.7truein}{
%\par\noindent
%This research involved lots of cool stuff.
%}

\medskip

\par\noindent
\textbf{\textit{Graduate Research}}\, The Pennsylvania State University\hfill 2011--2013
\par\noindent
Research Advisor: Advisor

\smallskip

\par\noindent
\hspace{0.10truein}  
\parbox{5.7truein}{
\par\noindent
Research Title
}

\medskip

\par\noindent
\textbf{\textit{Undergraduate Research}}\, Undergrad University\hfill 20XX--20XX
\par\noindent
Research Advisor: Undergrad Advisor

\smallskip

\par\noindent
\hspace{0.10truein}  
\parbox{5.7truein}{
\par\noindent
Undergrad Research Title
}

\medskip

\large
\centerline{{\bf Teaching Experience}}
\normalsize

%\smallskip
%
%\par\noindent
%\textbf{\textit{Guest Lecturer}}\, The Pennsylvania State University\hfill 199?--Present
%
%\smallskip
%
%\par\noindent
%\hspace{0.10truein}  
%\parbox{5.7truein}{
%\par\noindent
%I taught lectures which involved doing cool stuff.
%}

\medskip

\par\noindent
\textbf{\textit{Teaching Assistant}}\, The Pennsylvania State University\hfill Spring 20XX

\smallskip

\par\noindent
\hspace{0.10truein}  
\parbox{5.7truein}{
\par\noindent
Lab instructor for some class.
}

\medskip

\par\noindent
\textbf{\textit{Tutor}}\, Undergrad University\hfill 20XX-20XX

\smallskip

\par\noindent
\hspace{0.10truein}  
\parbox{5.7truein}{
\par\noindent
Individual tutor for students involved in undergraduate courses.
}
\end{singlespace}
}
\end{document}

%%% Local Variables: 
%%% mode: latex
%%% TeX-master: t
%%% End: 
