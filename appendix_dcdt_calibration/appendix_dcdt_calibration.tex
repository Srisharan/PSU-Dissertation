\chapter*{Appendix F. DCDT Calibration Procedure}
Electromagnetic signals have been reported in association with geophysical phenomena including earthquakes, landslides, and volcanic events. Mechanisms that suggested to explain seismoelectrical signals include triboelectricity, piezoelectricity, streaming potentials, and the migration
of electron holes, yet the origin of such phenomena remains poorly understood. We present results from laboratory experiments regarding the relationship between electrical and mechanical signals for frictional stick-slip events in sheared soda-lime glass bead layers. The results are interpreted in the context of lattice defect migration and granular force chain mechanics. During stick-slip events, we observe two distinct behaviors delineated by the attainment of a frictional stick-slip steady state. During initial shear loading, layers charge during stick-slip events and the potential of the system rises. After steady state stick-slip behavior is attained, the system begins to discharge. Coseismic signals are characterized by potential drops superimposed on a longer-term trend. We suggest that the observed signal is a convolution of two effects: charging of the forcing blocks and signals associated with the stress state of the material. The long-term charging of the blocks is accomplished by grain boundary movement during the initial establishment of force chain networks. Short-term signals associated with stick-slip events may originate from produced electron holes. Applied to tectonic faults, our results suggest that electrical signals generated during frictional failure may provide a way to monitor stress and the onset of earthquake rupture. Potential changes could produce detectable signals that may forecast the early stages of failure, providing a modest warning of the event.
