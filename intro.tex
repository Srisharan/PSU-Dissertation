\chapter*{Introduction}

The movement of rocks relative to each other caused by tectonic strain accumulation results in zones of localized deformation. Strain energy is effectively released as slip, which is most efficiently accommodated in a relatively thin zone of movement oriented favorably with the regional stress field. These zones of slip host a variety of complex processes which are still poorly understood. The material properties of the rocks, and therefore the mineralogy, controls the frictional properties of the interface. Geochemical processes can move minerals around through dissolution-precipitation processes and even accommodate slow creep movement of the rock. Hydrologic properties of the system can serve to increase or decrease the fluid pressure on the fault and even determine if mechanisms such as dilation hardening are allowed to operate. The existing geologic system can introduce complications through many processes, including the roughening of the fault surface through subduction of seamounts and other structural processes. These slowly evolving geologic systems are transformed during seismic slip of meters/second into systems with rapid change and extreme conditions which are poorly understood. The transformation between the two is quite possibly the most important part of the process and is not able to be explored in the real world due to the distance and measurement limitations imposed by the geology.

Laboratory rock mechanics studies have been recreating portions of fault zone conditions and behavior on a very small scale (millimeters to meters), but have not been widely utilized by the broader earthquake community. Seismologists often have data that is spatially limited by the location of measured events and cannot resolve the details that are observed in the laboratory. Experimentalists are often focused on the minutia of the frictional dynamics of the system and are not able to connect their observations to the large-scale global experiment of plate tectonics. This disconnect calls for the realization that rock mechanics could and should be treated as experimental seismology. In this dissertation, I attempt to study the general mechanisms of slip mode determination and connect them to a single natural system. While that is a limited scope, I believe that more such works can begin to build the bridge between the experimental and observational fields that is desperately needed to understand this complex phenomena.

Friction, or the force resisting the movement of two surfaces relative to each other, is one of the key parameters that control the nucleation stage of earthquakes. In the case a two-block slider model, friction is defined as the ratio of shear stress to normal stress. Early models of friction used a single value of friction for materials at all velocities and conditions. For most Earth materials, this number is in the range 0.6-0.8. Early explorations of friction began to discover that friction is not a static quantity, but that it was different when stationary versus when in motion. These so-called static and dynamic friction values are often as complicated as many analyses get, but even they do not explain all the observations of careful experimentation. Observations of the velocity dependence of friction, time dependence of friction, and memory effects from past states all indicated the need for a more complex formulation. The rate-and-state friction framework captures many of these observations, but is at its core an empirical relation with arbitrary frictional parameters.

The rate-and-state friction relation allows friction to change as a function of the sliding velocity of the system (rate) and the history of the system (state). The parameters in the basic relation are the direct effect, $a$, the evolution effect, $b$, and the critical slip distance $D_c$ which are fully described in chapter 1. The state of the system is determined by a second equation, the state relation. Common state relations have the state as a function of either time or slip on the fault. This simple model can capture behaviors such as velocity strengthening/weakening, time dependent healing, and even a transition in behavior from stable to unstable. In recent years there have been works which extend the basic model to use multiple state variables, designer friction relations in which the frictional parameters are themselves functions of velocity, and have introduced more complex state relations that depend on stressing rate or other parameters.

Analysis of the rate-and-state friction relation shows that a Hopf bifurcation occurs under certain conditions and completely changes the behavior of the system. On one side of the bifurcation, the system is stable and responds to perturbations in a finite time and the system reaches a new state of equilibrium in which friction is constant. The other side of the bifurcation represents unstable behavior. In the unstable regime, the system never reaches a new equilibrium state and is continually experiencing accelerations. Unstable systems nearly halt movement for some period of time. During this stuck period the far field movement of the system accumulates energy stored as strain in the elastic media. Once the friction on the interface reaches a critical value, the system catastrophically fails and rapidly accelerates. During the slip phase of motion, heat is generated, material is pulverized, and elastic radiation is produced. This stick-slip cycle has often been cited as an analog for naturally occurring earthquakes, as well as a variety of other phenomena including cutting tool chatter and animal sound production.

In terms of the rate-and-state friction framework, the bifurcation is observed to occur at a value of stiffness of the system termed the critical stiffness (kc). When the system is stiffer than the critical stiffness, energy can be released from the system faster than the system can weaken, so it remains in a stable sliding state. When the system is more compliant than the critical stiffness, the system weakens faster than energy can be released, resulting in a growing energy imbalance that produces large accelerations and stick-slip behavior.  

While the strict bifurcation of behavior seemed to be adequate for earthquake science - faults either hosted earthquakes or did not - the engineering sciences were not satisfied. Examination of other dynamical systems shows that there are often regions of transitory or intermediate behavior. The boundary between the edges may indeed be sharp, but not infinitely sharp as predicted. For example, the behavior of control systems shows that regions of critical stability exist in which the system is not completely unstable and uncontrolled, but in which the controller is struggling to control the process variable. Likewise, with more comprehensive physics or complex relations, the rate-and-state friction relation can begin to exhibit more complex behaviors. 

The discovery of slow earthquakes confirmed that the frictional behavior of faults is not a simple stable-unstable boundary, but in fact a continuum that hosts a rich variety of complex behaviors. In this dissertation, I use small-scale laboratory studies to explore this spectrum of behaviors and determine which frictional properties and complications are needed to explain the observations. I apply these ideas to a natural slow-slipping system beneath Whillans Ice Stream in western Antarctica. My main contributions are demonstrations of how to generate the range of slip modes in the laboratory, to show that the critical stiffness ratio indeed explains many of the first order behaviors, and to show that the behavior of a system is indeed velocity dependent.
